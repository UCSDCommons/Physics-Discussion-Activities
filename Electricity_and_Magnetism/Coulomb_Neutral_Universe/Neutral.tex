\documentclass{article}

\usepackage[margin=0.5in]{geometry}

\title{Is the universe electrically neutral on large scales?}
\author{Original Author: Nathan Butcher}
\date{}


\begin{document}
\maketitle
\thispagestyle{empty}

\textbf{Purpose of activity:} Start the class by using a simple principle (Coulomb's law) to reach a significant and interesting conclusion.

\textbf{Topic and objective:} This should be the first discussion section, ideally the first practice problem or example you do. Coulomb's law should be introduced at the very beginning of the class.

\textbf{Preparation:} Need to prepare 3-4 slides for the derivation or notes for a chalkboard derivation.

\textbf{Duration:} 10 minutes 

\textbf{Considerations and suggestions:}

\hspace{14pt}

\textbf{Activity:}
\begin{enumerate}
\item Start by asking the class ``Is the universe electrically neutral on large scales? How do you think we could test it?" Responses should be interesting and varied.
\item State that we can compare the force due to gravity and the force due to electric charge. Write out Newton's law of universal gravitation (explain what the variables are) and ask how it compares to Coulomb's law. The two key points are the $1/r^2$ dependence and the product of masses vs. product of charges.
\item Have the class group up into groups of 2-4 students. Ask them to find the ratio of the gravitational force to the Coulomb force between two protons at some distance. Give the class the constants
\end{enumerate}

\textbf{Reflection:} Questions or ideas for the students to think over or discuss that should help them focus on the key points.

\hspace{14pt}

Figures should be provided as a separate file in the same directory and be referenced in text.

\end{document}