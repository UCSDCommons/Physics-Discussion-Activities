\documentclass{article}

\usepackage[margin=0.5in]{geometry}

\title{Different cases of elastic collisions}
\author{Original Author: Nathan Butcher}
\date{}


\begin{document}
\maketitle
\thispagestyle{empty}

\textbf{Purpose of activity:} Have students think about elastic collisions in a qualitative manner and draw some general conclusions. Include a demo at the end to test their answers.

\textbf{Topic and objective:} Elastic collisions, can be done as soon as the topic is introduced in lecture. The objective is for students to apply conservation equations using variables for mass and velocity to draw conclusions about the final state of a system. 

\textbf{Preparation:} You will need to reserve the track and carts. Make sure you test the demo equipment before running it in the classroom. 

\textbf{Duration:} 15 to 20 minutes 

\textbf{Considerations and suggestions:} Check if the instructor did a similar comparison demo in class.

\hspace{14pt}

\textbf{Activity:}
\begin{enumerate}
\item Start by showing the image collision\_setup.png and stating that shows the initial state prior to a completely elastic collision. Note the arrow defining positive x to the right. Ask the class to individually write down a definition for elastic collisions. After one minute, have students pair up and share their definitions. Then go over the definition as a class, making sure to include
\begin{itemize}
	\item Momentum is conserved.
	\item Kinetic energy is conserved.
	\item The objects do not stick together (can have different final velocities).
\end{itemize}

\item The main work for students will be filling out the following table for post-collision information about the two objects

\hspace{12pt}

\begin{tabular}{| c | c | c |}
	\hline
	Case & Direction $m_1$ moves after collision? & Compare $|v_{2,f}|$ to $|v_{1,i}|$ \\ \hline
	$m_1 > m_2$ & & \\ \hline
	$m_1 < m_2$ & & \\ \hline 
	$m_1 = m_2$ & & \\ \hline
\end{tabular}

\hspace{12pt}

For the left column, students will be giving a direction as right, left, say it is stationary. For the right column, students will compare the speed of $m_2$ after the collision to $m_1$ before the collision.

\item Explain the table (maybe with example inputs to show what you are expecting) and have the class work in groups of 2 or 3 to fill in the entries. This could take 7-10 minutes, depending on the class. While they are working on it, be moving around the room ready to answer questions. After the first 2 or 3 minutes, start talking to groups about how they are doing with the activity.

\item Once you can see that the class is finished, it is time to fill in the table. I recommend drawing it on the board while they work and asking for volunteers to start filling in boxes. Ask for the students who fill in a box to state how they got their answer.

\item Show the derivation on the board or with lecture slides. An outline for a possible derivation is provided in the file "velocity\_derivation.pdf" that can be used to prepare your notes.

\item Use weighted carts on the low friction track to do a demo and demonstrate the calculated results.
\end{enumerate}

\textbf{Reflection:} Ask the class how the table would be different if the collision was perfectly inelastic. Have them discuss it in groups of 2 or 3 for a couple minutes and then ask for answers. The key points are

\begin{itemize}
	\item The objects will stick together, limiting what our answers can be.
	\item $m_1$ will move right in all cases since the stuck-together masses must move right to conserve momentum.
	\item $v_{2,f} < v_{1,i}$ in all cases because the final mass is larger than the initial mass. Therefore, the only way to conserve momentum is if the final velocity is lower.
\end{itemize}

\hspace{14pt}



\end{document}