\documentclass{article}

\usepackage[margin=0.5in]{geometry}

\title{Projectile motion by playing catch}
\author{Original Author: Nathan Butcher}
\date{6/25/2018}


\begin{document}
\maketitle
\thispagestyle{empty}

\textbf{Purpose of activity:}  Everybody has an intuitive idea of projectile motion from experience, so this lesson tries to draw on that to reinforce the mathematics of projectile motion.

\textbf{Topic and objective:} This lesson should follow the introduction of projectile motion. After this activity students should be able to qualitatively describe projectile motion using accurate terminology from the course. 

\textbf{Preparation:} You need a ball to play catch with (I used an American football) and should have a plan of questions for the class. 

\textbf{Duration:} 10 minutes 

\textbf{Considerations and suggestions:}

\hspace{14pt}

\textbf{Activity:}
\begin{enumerate}
\item Start by asking for a volunteer. It is up to you if you want to reveal anything about the demo before asking for a volunteer. I prefer not to (keeping the football in my bag until the volunteer has gotten to the front).
\item Start with an initial question
\item Lay out steps until complete
\end{enumerate}

\textbf{Reflection:} Questions or ideas for the students to think over or discuss that should help them focus on the key points.

\hspace{14pt}

Figures should be provided as a separate file in the same directory and be referenced in text.

\end{document}