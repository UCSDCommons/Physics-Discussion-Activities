\documentclass{article}

\usepackage[margin=0.5in]{geometry}

\title{Golfing on the Moon}
\author{Original Author: Nathan Butcher}
\date{}


\begin{document}
\maketitle
\thispagestyle{empty}

\textbf{Purpose of activity:} A practice calculation of projectile motion. By setting it on the moon we can ignore air resistance and force the consideration of a different gravitational constant.

\textbf{Topic and objective:} This should be done right after projectile motion is introduced. Students should leave this discussion section more comfortable with projectile motion calculations, as this will be a practice calculation with guidance. 

\textbf{Preparation:} You may wish to show a video. Prepare to go throw the calculations either with slides or a board calculation.

\textbf{Duration:} 15 minutes

\textbf{Considerations and suggestions:}

\hspace{14pt}

\textbf{Activity:}
\begin{enumerate}
\item Start with the background. Astronaut Alan Shepard, commander of the Apollo 14 mission, brought to the moon a 6 iron and some golf balls. We will try to to calculate some properties of the golf ball in flight.

\item Let's estimate that his shot had an initial speed of 20.0 m/s and launched at a $40.0^{\circ}$ angle. Note that the surface gravity of the moon is 1/6 that of Earth.

\item Ask the class to pair up and start with the following questions

\begin{enumerate}
	\item How far did the ball travel in meters? Keep 3 significant figures in your answer.
	\item Alan estimates he hit his shot over 200 yards. Is our calculation consistent with his guess? 1 inch = 2.54 cm and 1 yard = 36 inches.
\end{enumerate}

\item Give the class 5 minutes and then go over the solutions. Preferably you will ask for students to volunteer their groups' answer and describe their method. Answers are: (a) 246 m (b) 246 m = 269 yards, so his estimate is good.

\item Now we will continue onto a couple more questions

\begin{enumerate}
	\item How long was the ball in flight for?
	\item What other angle would allow the ball to fly the same distance with the same initial speed? Without doing the calculation, would the time of flight for this launch angle be greater than or less than the time of flight for the launch angle of $40.0^{\circ}$? Explain.
\end{enumerate}

\item Go over these solutions, and try to use questions to guide students' thinking on the time of flight comparison. Answers (a) 16 seconds (b) $50.0^{\circ}$, and this would have a longer time of flight than a launch angle of $40.0^{\circ}$.
\end{enumerate}

\textbf{Reflection:} Projectile motion is simply two independent kinematics equations that share some quantities (time, initial velocity, launch angle).

\hspace{14pt}


\end{document}