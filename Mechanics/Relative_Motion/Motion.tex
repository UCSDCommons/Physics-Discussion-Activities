\documentclass{article}

\usepackage[margin=0.5in]{geometry}

\title{Relative motion of students walking}
\author{Original Author: Nathan Butcher, Idea: Alex Meill}
\date{}


\begin{document}
\maketitle
\thispagestyle{empty}

\textbf{Purpose of activity:} Practice plotting motion by considering the perspective of different observers.

\textbf{Topic and objective:} This should be done in the first discussion to give students a chance to describe positions and velocities. Will also allow students to make simple graphs. 

\textbf{Preparation:} You may want to make the table for the graphs as a slide.

\textbf{Duration:} 15 minutes

\textbf{Considerations and suggestions:}

\hspace{14pt}

\textbf{Activity:}
\begin{enumerate}
\item Start by asking for 3 volunteers for a demo. Once they are all at the front, ask them to stand in a line so that they are "diagonal" to the seats (see provided Diagram.png).
\item Either put up a slide with the blank table or draw it on the board. In this table, the horizontal rows will indicate who the observer is and the vertical columns will be who is moving. There is a diagram (Diagram.png) where the students are indicated by colored dots and that is how the provided table is labeled

\begin{tabular}{| c | c | c | c |}
	\hline
	 & ...blue's position vs. time & ...green's position vs. time & ...red's position vs. time \\ \hline
	 Blue sees... & & & \\ \hline
	 Green sees... & & & \\ \hline 
	 Red sees... & & & \\ \hline
\end{tabular}

\item Ask the volunteers to move as the arrows in the diagram ask (so the middle student is stationary). Once they do this, ask the class to draw simple position vs. time for every entry of the table. 

\item After 6-8 minutes to work on this, ask for volunteers to draw plots on the board to fill in the table.
\end{enumerate}

\textbf{Reflection:} Questions or ideas for the students to think over or discuss that should help them focus on the key points.

\hspace{14pt}

Figures should be provided as a separate file in the same directory and be referenced in text.

\end{document}